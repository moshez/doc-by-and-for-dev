\usepackage{hyperref}
\usepackage{graphicx}
\usepackage{listings}
\usepackage{textcomp}
\usepackage{fancyvrb}

\title{Documentation By, and For, Developers}
\author{Moshe Zadka -- https://cobordism.com}
\date{2020}

\begin{document}
\begin{titlepage}
\maketitle
\end{titlepage}

\section{Welcome}

\frame{\titlepage}

\begin{frame}
\frametitle{Acknowledgement of Country}

San Francisco Bay Area Peninsula

Ancestral home of the Ramaytush Ohlone

\end{frame}

\section{Who?}

\begin{frame}
\frametitle{Who Are the Docs For?}

If nobody is reading them,
there is no reason to write them.
\end{frame}

\begin{frame}
\frametitle{Example: Configuration Management}

\begin{itemize}
\item Ansible
\item Salt
\item Puppet
\end{itemize}

\end{frame}


\begin{frame}
\frametitle{You}

Look out for number 1!
\end{frame}


\begin{frame}
\frametitle{Team}

Context

\end{frame}

\begin{frame}
\frametitle{Contributors}

People working on the core \pause (for first time)
\end{frame}

\begin{frame}
\frametitle{Developers}

People building on the code \pause plugins
\end{frame}

\begin{frame}
\frametitle{Users}

\begin{itemize}
\item Writing YAML files
\item Debugging YAML files
\end{itemize}

\end{frame}

\begin{frame}
\frametitle{Hats, not People}

Everyone plays multiple roles!
\end{frame}

\begin{frame}
\frametitle{Personas}

\begin{itemize}
\item Specific
\item Concrete
\end{itemize}

\end{frame}

\begin{frame}
\frametitle{What Do They Know?}

(And remember)
\end{frame}

\begin{frame}
\frametitle{What Do They Want?}

Not bedside reading
\end{frame}

\begin{frame}
\frametitle{What Are They Missing?}

Next step
\end{frame}

\begin{frame}
\frametitle{What Are They Looking For?}

\begin{itemize}
\item Keywords
\items Search terms
\end{itemize}
\end{frame}

\section{Kinds of Documentation}

\begin{frame}
\frametitle{Quickstart}

Generic guide

\end{frame}

\begin{frame}
\frametitle{Quickstart: Good For}

Making something work
\end{frame}


\begin{frame}
\frametitle{Quickstart: Bad For}

Troubleshooting
\end{frame}


\begin{frame}
\frametitle{Tutorial}

A specific path
\end{frame}


\begin{frame}
\frametitle{Tutorial: Good For}

Common use-case
\end{frame}


\begin{frame}
\frametitle{Tutorial: Bad For}

Advanced use-cases
\end{frame}


\begin{frame}
\frametitle{Example}

Documented sample
\end{frame}


\begin{frame}
\frametitle{Example: Good For}

Copy-paste
\end{frame}


\begin{frame}
\frametitle{Example: Bad For}

Non-specific use case
\end{frame}


\begin{frame}
\frametitle{Reference}

What does each thing mean/do
\end{frame}


\begin{frame}
\frametitle{Reference: Good For}

Answering specific questions

Troubleshooting
\end{frame}


\begin{frame}
\frametitle{Reference: Bad For}
Getting started fast
\end{frame}

\begin{frame}
\frametitle{FAQ}
Actually F
\end{frame}

\begin{frame}
\frametitle{FAQ: Good For}

Common stumbling blocks
\end{frame}

\begin{frame}
\frametitle{FAQ: Bad For}
Advocacy
\end{frame}


\section{Beyond Prose}


\begin{frame}
\frametitle{Screenshots}

RoI
\end{frame}

\begin{frame}
\frametitle{Diagrams}

\begin{itemize}
\item Great: Relationships
\item Good: Structure
\item Bad: Control flow
\end{itemize}


\end{frame}


\begin{frame}
\frametitle{Tables}
\end{frame}

\begin{frame}
\frametitle{Lists}

\begin{itemize}
\item Bullet
\item Number
\item Title
\end{itemize}

\end{frame}

\begin{frame}
\frametitle{Examples}

Self-contained vs. Succint
\end{frame}

\section{Good Documentation}


\begin{frame}
\frametitle{Paying Rent}

Why is this sentence here?

Why is this word here?
?

Why is this word here?

\end{frame}

\begin{frame}
\frametitle{Feedback}

Robots don't read
\end{frame}


\begin{frame}
\frametitle{Accuracy}

Be concrete, be correct

\end{frame}

\begin{frame}
\frametitle{Redundant}

Do Repeat Yourself
\end{frame}


\begin{frame}
\frametitle{Organized}

Order

Cohesion
\end{frame}

\begin{frame}
\frametitle{Clear Titles}

Is this what I need?
\end{frame}


\begin{frame}
\frametitle{Learning Flow}

Now, draw the rest of the owl
\end{frame}


\begin{frame}
\frametitle{Unambiguous}

Technical terms are useful
\end{frame}

\begin{frame}
\frametitle{Keep Scope}

Who is this for?

What are they trying to do?
\end{frame}

\section{Avoid}

\begin{frame}
\frametitle{What Not To Do}

Comic relief
\end{frame}

\begin{frame}
\frametitle{Fancy Words}

Thusly, this API could be utilized in the denounment of your function...
\end{frame}

\begin{frame}
\frametitle{Assumptions}

You already know...
\end{frame}

\begin{frame}
\frametitle{Insult}

Simply, all you need to to do is just pass a prime argument.
Obviously, Mersenne primes are not a good idea here...
\end{frame}

\begin{frame}[fragile]
\frametitle{Verbose}

\begin{lstlisting}
count:  How long to count for.
This should be a number between 1 and 5.
For example, 6 would not be a valid argument.
Neither would 0.
The number 9 is right out!
\end{lstlisting}
\end{frame}

\begin{frame}[fragile]
\frametitle{Terse}

\begin{lstlisting}
count: The count
\end{lstlisting}
\end{frame}

\begin{frame}
\frametitle{Inexact words}

\begin{lstlisting}
count: The bound of the count
\end{lstlisting}

\end{frame}

\begin{frame}
\frametitle{Tangents}

\begin{lstlisting}
count: Number between 1 and 5 to count to. The performance implications
here depend on exactly how you define big-O notation.
In general, there are several ways to define big-O notation,
which are not completely equivalent.
\end{lstlisting}

\end{frame}

\begin{frame}
\frametitle{Rationalize}
\begin{lstlisting}
count: A number from 1 to 5. Limiting the upper bound to 5 makes
the implementation much simpler, and few people need to be able
to count to higher numbers.
\end{lstlisting}

\end{frame}

\begin{frame}
\frametitle{Mislead}

\begin{lstlisting}
count: Any number is supported here.
\end{lstlisting}
\end{frame}

\section{Tooling}

\begin{frame}
\frametitle{Language Native}

\begin{itemize}
\item JavaDoc
\item Rust docs
\item Etc.
\end{itemize}

\end{frame}

\begin{frame}
\frametitle{Sphinx}

\begin{itemize}
\item Integration with (some) language native
\item Long-form prose
\item Semantic mark-up
\end{itemize}

\end{frame}

\begin{frame}
\frametitle{MkDocs}

\begin{itemize}
\item Easier to get started
\item Less functionality
\end{itemize}

\end{frame}

\begin{frame}
\frametitle{Continuous Integration}

\begin{itemize}
\item Minimum: Build
\item Medium: Lint/checks
\item Awesome: Generate with link
\end{itemize}

\end{frame}

\begin[frame}
\frametitle{Example Testing}

Partial line selection
\end{frame}


\begin{frame}
\frametitle{Development Flow}

Doc change in same patch?
\end{frame}

\begin{frame}
\frametitle{In or Out?}

\begin{itemize}
\item Reference
\item Tutorials
\item Guides
\item Who is authoring?
\end{itemize}

\end{frame}

\section{Conclusion}

\begin{frame}
\frametitle{Final Thoughts}

\begin{itemize}
\item Just do it
\item You'll suck
\item You'll get better
\end{itemize}

\end{frame}

\begin{frame}
\frametitle{Take Aways}

\begin{itemize}
\item Who?
\item What?
\item How?
\end{itemize}

\end{frame}

\end{document}
